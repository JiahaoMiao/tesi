\documentclass[10pt,english,openany]{book}

\usepackage{style/relazione}
\usepackage{style/mystyle}

% Adding the path for graphics
\graphicspath{{../figures/}}

% Setting the path for input files
\makeatletter
\def\input@path{{../}}
\makeatother

\addbibresource{bibliography/biblio.bib} % The file containing the bibliography

% BACHELOR DEGREE COURSE
\def\myCDL{Corso di Laurea triennale in Fisica}

% TITLE:
\def\myTitle{Resummation of QCD large infrared logarithms \\for Thrust distribution in \lowercase{$e^+e^-$} annihilation} 

% AUTHOR:
\def\myName{Jiahao Miao}
\def\myMat{Matr. Nr. 936136}
 
\def\myRefereeA{Giancarlo Ferrera}
\def\myRefereeB{Wan-Li Ju}
 
% ANNO ACCADEMICO
\def\myYY{2023-2024}
 
% This command introduces a list of figures after the table of contents (optional)
\figurespagetrue

% This command introduces a list of tables after the table of contents (optional)
\tablespagetrue


\begin{document}

% frontispiece with logo, title, author, referee, etc.
% \frontispiece

\chapter*{Summary}

In this thesis, we have studied the resummation of QCD large infrared logarithms for the Thrust distribution in $e^+e^-$ annihilation.
The Thrust distribution is a classical precision observable in collider physics and it allows the extraction of the strong coupling constant $\alpha_s$,
which is a key parameter of the Standard Model of particle physics and is essential for the precision tests of the theory and the search for new physics at the LHC.

In the era of the LEP experiments(1989-2000) the calculations for certain classes of variables, such as the Thrust event shape variable at next-to-leading order (NLO), 
were improved by resumming leading and next-to-leading logarithmic terms (NLL). These calculations, matched to fixed-order expressions, enlarged the
kinematic range of applicability for $\alpha_s$ extractions and reduced the systematic theoretical uncertainty. 

This thesis aims to enhance these achievements by incorporating higher-order logarithmic terms in the resummation process and utilizing NNLO fixed-order calculations completed in 2007, 
which were unavailable during the LEP experiments.

Resummation tecniques are needed because in the kinematical region where the Thrust $\tau$ is close to 0, cancellations between real and virtual corrections 
are not complete due to kinematical constraints and large logarithms of the form $\alpha_s^n \log^m(\frac{1}{\tau})$ with $m \le 2n$ appear to spoil the convergence of the perturbative series.
Therefore, resummation of these logarithms is needed to obtain a reliable theoretical prediction for the Thrust distribution in the region of small $\tau$, 
then matching the resummed prediction to the fixed order calculation yields a prediction that is valid in a larger range of $\tau$ than the fixed order calculation alone.
Resummed calculations allows us to push the validity of QCD perturbation theory to the boundary of the phase space
where fixed order predictions are not reliable.

Initially, we calculated the resummation functions $f_i(\lambda)$ in Laplace space up to $i=5$, where the natural exponentiation of the logarithms occurs.
We were careful in ensuring consistency with prior findings $i=1,2,3$ and extracting necessary ingredients for resummation up to N$^4$LL accuracy from existing literature.

Subsequently, we performed the inverse Laplace transform of the resummed distribution following methodologies outlined in the literature 
and extended the results beyond next-to-next-Leading Logarithm (NNLL) accuracy present in literature up to N$^4$LL accuracy. This has allowed us to obtain the resummed Thrust distribution in thrust space, whose shape 
closely resembles experimental data from LEP (ALEPH, OPAL, L3 and DELPHI experiment) and SLAC (LSD experiment).

We also utilized a stochastic optimation algorithm (genetic algorithm) to determine the kinematical lower limit of the Thrust
for N=5 partons final state, which is necessary for the correct normalization of the fixed-order cross section at NNLO.

To refine our theoretical predictions further, we have matched the resummed expression to the fixed-order NNLO calculations at large values of $\tau$ using the R-matching scheme.
The results presented in this thesis allow us to achieve NNLO+N$^3$LL accuracy.
This represents a step towards achieving a more precise theoretical prediction for the Thrust distribution in $e^+e^-$ annihilation and
a more precise extraction of the strong coupling constant $\alpha_s$.

\end{document}
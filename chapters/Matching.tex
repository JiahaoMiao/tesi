\makeatletter % Use \makeatletter to make '@' a letter
\def\input@path{{../}} % Define input path to parent directory
\makeatother % Restore default category code of '@'

\documentclass[../main.tex]{subfiles}

\begin{document}

\section{Matching of resummation to fixed-order calculations}\label{sec:Matching}

Having obtained a resummed expression such as \cref{eq:RT resummed} for the shape cross sections at small values
of $\tau$, one can now match the resummed expression to the fixed-order NNLO calculations at large values of $\tau$
using the log(R)-mathing scheme \cite{CATANI19933}. The results presented in the previous sections allow us to 
compare the predictions at N$^3$LL accuracy with the fixed-order calculations at NNLO.

In the log(R)-matching scheme, at N$^3$LL+NNLO accuracy the matching procedure is given by comparing 
the logarithm of the fixed order result \cref{eq:Fixed_order}:

\begin{equation}
    \log(R_T(\tau))= A(\tau) \bar{\alpha}_s + \qty( B(\tau) - \frac{1}{2}A^2(\tau)) \bar{\alpha}_s^2 + \qty(C(\tau) - A(\tau) B(\tau) + \frac{1}{3}A^3(\tau)) \bar{\alpha}_s^3 + \order{\bar{\alpha}_s^4}, 
\end{equation}

with the resummed expression \cref{eq:RT resummed} and subtracting the common logarithmic terms, one obtains the following expression:

\begin{flalign}\label{eq:matching}
        \log(R(\tau,Q)) &= L g_1(\lambda)+ g_2(\lambda)+ \alpha_s g_3(\lambda)+\alpha_s^2 g_4(\lambda)\\
        &+ \bar{\alpha_s} \qty( A(\tau) - G_{12}L^2 - G_{11}L) \nonumber\\
        &+ \bar{\alpha_s}^2 \qty( B(\tau) - \frac{1}{2}A^2(\tau) -G_{23}L^3 - G_{22}L^2 - G_{21}L ) \nonumber\\
        &+ \bar{\alpha_s}^3 \qty( C(\tau) - A(\tau) B(\tau) + \frac{1}{3}A^3(\tau) -G_{34}L^4 - G_{33}L^3 - G_{32}L^2 - G_{31}L )\nonumber
\end{flalign}



\end{document}
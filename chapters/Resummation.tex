\makeatletter % Use \makeatletter to make '@' a letter
\def\input@path{{../}} % Define input path to parent directory
\makeatother % Restore default category code of '@'

\documentclass[../main.tex]{subfiles}

\begin{document}

\section{Resummation}


\subsection{Soft-gluon effects in QCD cross sections}

Soft-gluon effects and the soft gluon exponentiation are reviewed in \cite{Catani_1997} and \cite{catani1997softgluon}, here i summarize the physical 
motivation and main idea of the resummation of soft gluon effects in QCD.

The finite energy resolution inherent in any particle detector implies that the physical cross sections, 
those experimentally measured, inherently incorporate all contributions from arbitrarily soft particles 
produced in the final state. In other words, because we lack the ability to precisely resolve the energy 
of soft particles, we are unable to distinguish between their presence or absence in our calculations. 
Consequently, we must account for the sum over all possible final states.

This inclusiveness is essential in QCD calculations. Higher order perturbative contributions due to \emph{virtual}
gluons are infrared divergent and the divergences are exactly  cancelled by radiation of undetected \emph{real} gluons.
speaking the cancellation does not necessarily take place \emph{order by order} in perturbative theory.
In particular kinematic configurations, \emph{e.g} Thrust in the dijet limit $T \to 1$ ,   
real and virtual contributions can be highly unbalanced, because the emission of real radiation is inhibited by kinematic constrainsts,
spoiling the cancellation mechanism. As a result, soft gluon contribution to QCD cross sections can still be either large or singular.

In these cases, the cancellation of infrared divergences bequeaths higher order contributions of the form:

\begin{equation}\label{eq:log_enhanced_terms}
    C_{nm} \alpha_s^n \ln^m\qty(\frac{1}{1-T}) , \qq{with $m \le 2n $,}
\end{equation}

that can become large, $\alpha_s \ln^2\qty(1-x)\lesssim 1$, even if the QCD coupling $\alpha_s$ is in the perturbative regime \\
$\alpha_s \ll 1$.
The logarithmically enhanced terms in \cref{eq:log_enhanced_terms} are certainly relevant near the dijet limit $T->1$.
In these cases, see \cite{CATANI19933},\cite{CATANI1991491}, the theoretical predictions can be improved by evaluating soft gluon 
contributions to high orders and possibly resumming to all of them in $\alpha_s$.

\subsection{Resummation of soft-gluon effects}

The physical bases for all order summation of soft gluon contributions to QCD cross sections are the following:
to QCD are dynamics and kinematics factorizations. The first factorization follows from gauge invariance and 
unitarity: in the soft limit multi-gluon amplitudes fulfil generalized factorization formulae given in terms of a single gluon emission
probability that is universal \emph{i.e} process independent. The second factorization regards kinematics and is strongly dependent on the observable to because
computed. 

\emph{If}, in the appropriate soft limit, the phase space for this observable can be written in a factorized way, resummation is feasible in form of 
generalized exponentiation of the single gluon emission probability. Then exponentiation allows one to dfine and carry out an improved perturbative expansion
that systematically resums $LL$, terms, $NLL$ terms and so on.

In general the phase space is not factorizable in single particle contributions. If when factorizable, it does not occur in the space where the physical observable 
x is defined. Usually, it is necessary to introduce a conjugate space where the physical observable x is defined. Usually, it is necessary to 
introduce a conjugate space to overcome phase-space constraints. 

A typical example is the energy conservation constraints that can be factorized by working in N-moment space, N being the variable conjugate to the energy x 
via a Mellin (or Laplace) transformation.

Large or Singular soft gluon contributions can have different origins and resummation takes different exponentiation forms depending on kinematics.
This leads to varieties of Sudakov effects.

One of the effects is Sudakov suppression: Soft gluon resummation produces suppression of cross sections near the exclusive phase space boundary.
Typical examples are $e^+e^-$ event shapes distributions like Thrust $T$. 

\end{document}
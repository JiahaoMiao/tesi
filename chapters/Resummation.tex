\makeatletter % Use \makeatletter to make '@' a letter
\def\input@path{{../}} % Define input path to parent directory
\makeatother % Restore default category code of '@'

\documentclass[../main.tex]{subfiles}

\begin{document}

\chapter{Resummation formalism}\label{ch:resummation}

\section{Soft-gluon effects in QCD cross sections} \label{sec:Soft_gluon_effects}

Soft-gluon effects and the soft-gluon exponentiation are reviewed in \cite{Catani_1997} and \cite{catani1997softgluon}, here I summarize the physical 
motivation and main ideas of the resummation of soft gluon effects in QCD.

The finite energy resolution in any particle detector implies that the physical cross sections, 
those experimentally measured, inherently incorporate all contributions from arbitrarily soft particles 
produced in the final state. In other words, because we lack the ability to precisely resolve the energy 
of soft particles, we are unable to distinguish between their presence or absence in our calculations (see \cref{fig:NLO virtual corrections}). 
Consequently, we must account for the sum over all possible final states.

\begin{figure}[h]
    \centering
    \begin{subfigure}[b]{0.45\textwidth}
        \centering
        \begin{tikzpicture}
            \begin{feynman}
                \vertex (a);
                \vertex [above left =of a] (i1) {$e^{-}$};
                \vertex [below left =of a] (i2) {$e^{+}$};
                \vertex (b) [right =of a];
                \vertex [above right =0.5cm of b] (f1);
                \vertex (c) [above right =0.5cm of b];
                \vertex (d) [above right =1cm of c];
                \vertex [above right =0.5cm of d] (f3) {$q$};
                \vertex [below right =2cm of b] (f2) {$\bar{q}$};
    
                \diagram* {
                    (i1) -- [fermion] (a) -- [fermion] (i2);
                    (a) -- [photon, edge label=\(\gamma^*\),momentum'= \(k\)] (b);
                    (f2) -- [fermion] (b) -- [edge label=\(q\)] (f1);
                    (c) -- [gluon, edge label'=$g$,half right] (d);
                    (c) -- [fermion] (d);
                    (d)-- (f3);
                };
            \end{feynman}
        \end{tikzpicture}
    \end{subfigure}
    \hfill
    \begin{subfigure}[b]{0.45\textwidth}
        \centering
        \begin{tikzpicture}
            \begin{feynman}
                \vertex (a);
                \vertex [above left =of a] (i1) {$e^{-}$};
                \vertex [below left =of a] (i2) {$e^{+}$};
                \vertex (b) [right =of a];
                \vertex [above right =of b] (c);
                \vertex [below right =of b] (d);
                \vertex [above right =0.75cm of c] (f1) {$q$};
                \vertex [below right =0.75cm of d] (f2) {$\bar{q}$};
    
                \diagram* {
                    (i1) -- [fermion] (a) -- [fermion] (i2);
                    (a) -- [photon, edge label=\(\gamma^*\),momentum'= \(k\)] (b);
                    (d) -- [edge label=\(\bar{q}\)] (b) -- [edge label=\(q\)] (c);
                    (c) -- [gluon, edge label'=$g$] (d);
                    (c) -- [fermion] (f1);
                    (d) -- [anti fermion] (f2);
                };
            \end{feynman}
        \end{tikzpicture}
    \end{subfigure}
    \caption{Examples of one loop Feynman diagrams whose final state is identical to the Tree level diagram in \cref{fig:ep_annihilation} } 
    \label{fig:NLO virtual corrections}
\end{figure}

This inclusiveness is essential in QCD calculations. Higher order perturbative contributions due to \emph{virtual}
gluons are infrared divergent and the divergences are exactly  cancelled by radiation of undetected \emph{real} gluons.
In particular kinematic configurations, \emph{e.g} Thrust in the dijet limit $T \to 1$ ,   
real and virtual contributions are highly unbalanced, because the emission of real radiation is inhibited by kinematic constrainsts,
spoiling the cancellation mechanism. As a result, soft gluon contribution to QCD cross sections can still be either large or singular.

In these cases, the cancellation of infrared divergences bequeaths higher order contributions of the form:

\begin{equation}\label{eq:log_enhanced_terms}
    G_{nm} \alpha_s^n \ln^m \frac{1}{\tau} , \qq{with $m \le 2n $,}
\end{equation}

that can become large, $\alpha_s \ln^2\frac{1}{\tau}\gtrsim 1$, even if the QCD coupling $\alpha_s$ is in the perturbative regime $\alpha_s \ll 1$.
These logarithmically enhanced terms in \cref{eq:log_enhanced_terms} are certainly relevant near the dijet limit $\tau \to 0$.
In these cases, the theoretical predictions can be improved by evaluating soft gluon 
contributions to high orders and possibly resumming to all of them in $\alpha_s$ \cite{CATANI1991491,Monni:2011gb}.

The resummation of large logarithms in event shape distributions was described by Catani, Trentadue, Turnock, and Webber (CTTW) \cite{CATANI19933}.

The physical basis for all-order resummation of soft-gluon contributions to QCD cross sections are dynamics and kinematics factorizations. The first factorization follows from 
gauge invariance and unitarity while the second factorization strongly depends on the observable to be computed.

In the appropriate soft limit, if the phase-space for this observable can be written in a factorized way, then resummation is feasible in the form of a generalized exponentiation \cite{Sterman:1986aj}.
However even when phase-space factorization is achievable, it does not always occur in the space where the physical observable $x$ is defined.

Thrust is a good example of this situation, in fact, the thrust distribution admits a factorization in Laplace space, 
where the observable is the Laplace transform of the thrust distribution. 


\section{CTTW formalism}\label{sec:CTTW convention}

According to general theorems \cite{Block:1937},\cite{Kinoshita:1962ur},\cite{Lee:1964is}, 
the cumulant cross section $R_T(\tau)$ \cref{eq:cumulative_distribution} has a power series expansion in $\alpha_s(Q^2)$ of the form:

\begin{equation}\label{eq:cumulant_expansion}
    R_T(\tau) =  C\qty(\alpha_s\qty(Q^2))\Sigma\qty(\tau,\alpha_s\qty(Q^2)) + D\qty(\tau,\alpha_s\qty(Q^2))
\end{equation}

where 

\begin{align}
    C\qty(\alpha_s) &= 1 + \sum_{n=1}^{\infty} C_n \bar{\alpha}_s^n , \\
    \Sigma\qty(\tau,\alpha_s)&= \exp[\sum_{n=1}^\infty \bar{\alpha}_s^n \sum_{m=1}^{2n}G_{nm}\ln^m \frac{1}{\tau}] \label{eq:sigma} ,\\
    D(\tau,\alpha_s) &= \sum_{n=1}^{\infty} \bar{\alpha}_s^n D_n(\tau).
\end{align}

Here the $C_n$ and $G_{nm}$ are constants and $\bar{\alpha}_s = \frac{\alpha_s}{2\pi}$, while $D_n(\tau)$ 
is the non-singular part of the fixed-order expansion of $R_T(\tau)$ \cref{eq:Fixed_order_R}.

Thus at small $\tau$ (large thrust) it becomes morst important to resum the series of large logarithms in $\Sigma\qty(\tau,\alpha_s)$.
These are normally classified as \emph{leading} logarithms when $n < m \le 2n$, \emph{next-to-leading} when $m = n$ 
and \emph{subdominant} logarithms when $m < n$. 

The cumulant cross section $R(\tau)$ in the small $\tau$ region, in general, can be written in an exponential form as : (neglecting the $D(\tau,\alpha_s)$ term)

\begin{equation}\label{eq:resummed_R}
    R_T(\tau) \simeq \qty(1 + \sum_{n=1}^{\infty} C_n \bar{\alpha_s}^n)\exp[L g_1(\lambda)  + g_2(\lambda) + g_3(\lambda) \alpha_s + g_4(\lambda) \alpha_s^2 + g_5(\lambda) \alpha_s^3 + \order{\alpha_s^4}]
\end{equation}

where $L = \ln \frac{1}{\tau}$ and $\lambda = \alpha_s b_0 L$. The function $g_1$ encodes all the leading logarithms, the function $g_2$
resums all next-to-leading logarithms and so on.

The last equation gives a better prediction of the thrust distribution in the two-jet
region, but fails to describe the multijet region $\tau \to \tau_{max}$ , where non-singular pieces of the
fixed-order prediction become important. To achieve a reliable description of the observable
over a broader kinematical range the two expressions \cref{eq:Fixed_order_R} and \cref{eq:cumulant_expansion} can be matched,
taking care of double counting of logarithms appearing in both expressions.

Expanding \cref{eq:cumulant_expansion} in powers of $\alpha_s$ we have:

\begin{equation}
    R(\tau) = 1 + R^{(1)}(\tau) \bar{\alpha}_s + R^{(2)}(\tau)\bar{\alpha}_s^2 + R^{(3)}(\tau)\bar{\alpha}_s^3+ \dots
\end{equation}

where 

\begin{equation}
    R^{(n)}(\tau) = \sum_{m=1}^{2n} R_{nm} \ln^m \frac{1}{\tau} + D_n(\tau)
\end{equation}

with the first three terms given by:

\begin{equation}
    \begin{split}
        R^{(1)}(\tau) &= C_1  + G_{12} \ln ^2\frac{1}{\tau}+G_{11} \ln \frac{1}{\tau} + D_1(\tau)\\
        R^{(2)}(\tau) & = C_2 +\frac{1}{2} G_{12}^2 \ln ^4\frac{1}{\tau} +\left(G_{11} G_{12}+G_{23}\right) \ln ^3\frac{1}{\tau}\\
        &+\ln ^2\frac{1}{\tau} \Bigl(C_1 G_{12}+\frac{G_{11}^2}{2}+G_{22}\Bigr)+\ln \frac{1}{\tau} \Bigl(C_1 G_{11}+G_{21}\Bigr)\\
        &+ D_2(\tau)\\
        R^{(3)}(\tau) &= C_3 +\frac{1}{6} G_{12}^3 \ln ^6\frac{1}{\tau}+\left(\frac{1}{2} G_{11} G_{12}^2+G_{23} G_{12}\right) \ln ^5\frac{1}{\tau} \\
        &+ \ln ^4\frac{1}{\tau} \left(\frac{1}{2} C_1 G_{12}^2+\frac{1}{2} G_{12} G_{11}^2+G_{23} G_{11}+G_{12} G_{22}+G_{34}\right)\\
        &+\ln ^3\frac{1}{\tau} \left(C_1 G_{12} G_{11}+C_1 G_{23}+\frac{G_{11}^3}{6}+G_{22} G_{11}+G_{12} G_{21}+G_{33}\right)\\
        &+\ln ^2\frac{1}{\tau} \left(\frac{1}{2} C_1 G_{11}^2+C_2 G_{12}+C_1 G_{22}+G_{21} G_{11}+G_{32}\right)\\
        &+\ln \frac{1}{\tau} \left(C_2 G_{11}+C_1 G_{21}+G_{31}\right) + D_3(\tau)
    \end{split}
\end{equation}

To see how the resummation works, we can write the contributions to the cross section in a tabular form:

\begin{table}[h]
    \centering
    \caption{Fixed-order and resummed contributions to the cross section.}

    \begin{tabular}{c c c c c c c c}
        &&\(\text{LL}\) & \(\text{NLL}\) &\(\text{NNLL}\) & \(\text{N$^3$LL}\) & \(\text{N$^4$LL}\) & \(\cdots\)\\
        \(\text{LO}\)&\(\bar{\alpha}_s^1\) &\(R_{12}\) & \(R_{11}\) & \(C_1+D_1(\tau)\) & -  & - &  - \\
        \(\text{NLO}\)&\(\bar{\alpha}_s^2\) & \(R_{23}\) & \(R_{22}\) & \(R_{21}\) & \(C_2 + D_2(\tau)\) & - & -  \\
        \(\text{NNLO}\)&\(\bar{\alpha}_s^3\) & \(R_{34}\) & \(R_{33}\) & \(R_{32}\) & \(R_{31}\) & \(C_3 + D_3(\tau)\) & -  \\
        \(\text{N$^3$LO}\)&\(\bar{\alpha}_s^4\) & \(R_{45}\) & \(R_{44}\) & \(R_{43}\) & \(R_{42}\) & \(R_{41}\) & \(\dots\) \\
        & \(\vdots\) & \(\vdots\) & \(\vdots\) & \(\vdots\) & \(\vdots\) & \(\vdots\) & \(\ddots\) 
    \end{tabular}
      
\end{table}

Normally, in perturbation theory, the fixed-order expansion of the cross section is expanded in powers of $\alpha_s$, 
line by line in the table above thereby taking all $\alpha_s$ terms first (the so called \emph{leading} order, LO), then all $\alpha_s^2$ terms (the \emph{next-to-leading} order, NLO) and so on.
However, for small $\tau$ one has that $\alpha_s L\sim 1$ with $L = \ln \frac{1}{\tau}$, therefore by only taking the first line one does not
take into account all other terms in the first column which are all of the same magnitude. 
In order to have a better prediction of the cross section in the small $\tau$ region, one needs to sum all the terms in the first column first (the \emph{leading logarithms}, LL),
then all the terms in the second column (the \emph{next-to-leading logarithms}, NLL) and so on.


This is achieved by knowing the resummation coefficients $g_i$ in \cref{eq:resummed_R}. In fact by expanding $\ln\frac{1}{\tau} g_1(\lambda)$ in $\alpha_s$ we get all the coefficients $G_{n\space n+1}$, expanding
$g_2(\lambda)$ yields the coefficients $G_{n\space n}$, expading $\alpha_s g_3(\lambda) $ yields $G_{n \space n-1}$ and so on.

The difference between the logarithmic part and the full fixed-order series at different orders is given by the remainder function $D(\tau,\alpha_s)$:

\begin{equation}
    \begin{split}
        D_1(\tau) &=  A(\tau) - R^{(1)}(\tau)\\
        D_2(\tau) &=  B(\tau) - R^{(2)}(\tau)\\
        D_3(\tau) &=  C(\tau) - R^{(3)}(\tau)
    \end{split}
\end{equation}

which contains the non-logarithmic part of the fixed-order contribution and vanish for $\tau \to 0$.

However, in order to calculate the resummation coefficients $g_i(\lambda)$, whose expansion in $\alpha_s$ yields the $G_{nm}$ coefficients,  we need to perform an inverse transformation from Laplace space to real space.
This necessity arises because the factorization is carried out in Laplace space. As demonstrated in \cite{CATANI19933}, the problem of resummation
can be recast in the form of an integral equation in Laplace space \cref{eq:master formula},  whose solution directly provides the exponent function.

\begin{equation}\label{eq:master formula}
    \ln \tilde{J}_\nu^q(Q^2) = \int_0^1 \frac{\dd u}{u} \qty(e^{-u \nu Q^2}-1)\qty[ \int_{u^2 Q^2}^{u Q^2} \frac{1}{q^2} A\qty(\alpha_s(q^2)) \dd q^2 + \frac{1}{2} B\qty(\alpha_s(u Q^2))]
\end{equation}

where $\tilde{J}_\nu^q(Q^2)$ is the Laplace transform of the quark jet mass distribution, a quantity related to the thrust distribution.

We'll see explicitly in the next chapter that resolving the above integral gives the exponent:

\begin{equation}\label{eq:exponent F}
    \mathcal{F}(\alpha_s,\ln N) = L f_1(\lambda)  + f_2(\lambda) + f_3(\lambda) \alpha_s + f_4(\lambda) \alpha_s^2 + f_5(\lambda) \alpha_s^3 + \order{\alpha_s^4}
\end{equation}

where $L = \ln N = \ln(\nu Q^2)$ and $\lambda = \alpha_s b_0 L$.
We require the functions $f_i(\lambda)$ to be omogeneous,\emph{i.e} $f_i(0)=0$, so that at N$^n$LL we can write:

\begin{equation}
    f_{n+1}(\lambda) = \sum_{k \ge n} \tilde{G}_{k,k+1-n} \alpha_s^k L^{k+1-n}
\end{equation}

this requirement is automatically satisfied if we choose as variable $L = \ln(\frac{N}{N_0})$ where $N_0 = e^{-\gamma_E}$, $\gamma_E = 0.5772 \dots $ being the Euler-Mascheroni constant.
With the latter choice the terms proportional to $\gamma_E$ and its powers disappear. The advantage of the variable N is that the total rate is directly reproduced by setting $N=1$, while
in the variable $n=N/N_0$ it's when $N=N_0$. These two choices differ only by terms of higher order in $\gamma_E$.
In literature the variable $N$ is more common and in order to compare with other results we'll use the variable $N$, we can recover the results in the variable $n$ by simply setting $\gamma_E = 0$.

In the presentation of CTTW, N$^k$LL accuracy means how many terms $g_i$ in the exponent of \cref{eq:resummed_R} are known.
\end{document}
\makeatletter % Use \makeatletter to make '@' a letter
\def\input@path{{../}} % Define input path to parent directory
\makeatother % Restore default category code of '@'

\documentclass[../main.tex]{subfiles}

\begin{document}

% \Chapter{Calculations}\label{ch:calculations}

\section{Calculation of the $f_i$ functions}\label{sec:f functions}

In the two-jet region the fixed-order thrust distribution is enhanced by large infrared
logarithms which spoil the convergence of the perturbative series. The convergence can be
restored by resumming the logarithms to all orders in the coupling constant $\alpha_s$.

According to general theorems \cite{Block:1937},\cite{Kinoshita:1962ur},\cite{Lee:1964is}, 
the cross section \cref{eq:cross_section} has a power series expansion in $\alpha_s(Q^2)$ of the form:

\begin{equation}
    \frac{\sigma(\tau,Q^2)}{\sigma_t} =  C\qty(\alpha_s\qty(Q^2))\Sigma\qty(\tau,\alpha_s\qty(Q^2))+F\qty(\tau,\alpha_s\qty(Q^2))
\end{equation}

where $\sigma_t$ is the total hadronic cross section and    

\begin{subequations}
\begin{align}
    C\qty(\alpha_s) &= 1 + \sum_{n=1}^{\infty} C_n \alpha_s^n \\
    \Sigma\qty(\tau,\alpha_s)&= \exp[\sum_{n=1}^\infty \alpha_s^n \sum_{m=1}^{2n}G_{nm}\ln^m\tau] \\
    F(\tau,\alpha_s) &= \sum_{n=1}^{\infty} \alpha_s^n F_n(\tau)
\end{align}
\end{subequations}

Here $C_n$ and $G_{nm}$ are constants, while $F_n(\tau)$ are perturbatively computable functions that vanish at small $\tau$.
Thus at small $\tau$ (large thrust) it becomes morst important to resum the series of large logarithms in $\Sigma\qty(\tau,\alpha_s)$.
These are normally classified as \emph{leading} logarithms when $n < m \le 2n$, \emph{next-to-leading} when $m = n$ 
and \emph{subdominant} logarithms when $m < n$. 

In the article by Catani, Turnock, Webber and Trentadue \cite{CATANI1991491}, it was observed that 
for a final state configuration corresponding to a large value of thrust , \cref{eq:Thrust} can be approximated by

\begin{equation}
    \tau = 1-T \approx \frac{k_1^2+k_2^2}{Q^2}
\end{equation}

where $k_1^2$ and $k_2^2$ are the invariant masses squared of two back-to-back jets and $Q^2$ is the energy of the center of mass.
Thus the key to the evaluation of the thrust distributions is its relation to the quark jet mass distribution $J^q(Q^2,k^2)$ which denotes
the probability of jet invariant mass-squared $k^2$ at scale $Q^2$, then the thrust fraction 

\begin{equation}
    R(\tau,\alpha_s(Q^2)) = \frac{\sigma(\tau,Q^2)}{\sigma_t} = \frac{1}{\sigma_t} 
    \int_0^1 \dv{\sigma(\tau',Q^2)}{\tau'} \Theta (\tau - \tau') \dd \tau'
\end{equation}

takes the form of a convolution of two jet mass distributions $J(Q^2,k_1^2)$ and $J(Q^2,k_2^2)$

\begin{equation}
    R(\tau,\alpha_s(Q^2)) \underset{\tau \ll 1}{=} \int_0^\infty J(Q^2,k_1^2) J(Q^2,k_2^2) \Theta\qty(\tau - \frac{k_1^2+k_2^2}{Q^2}) \dd k_1^2 \dd k_2^2
\end{equation}

Introducing the Laplace transform of the jet mass distribution

\begin{equation}\label{eq:laplace_jet_mass}
    \tilde{J}^q_\nu(Q^2) = \int_0^\infty J^q(Q^2,k^2) e^{-\nu k^2} \dd k^2 
\end{equation}

and using the integral representation of the Heaviside step function

\begin{equation} \label{eq:heaviside_integral_representation}
    \Theta(\tau - \frac{k^2}{Q^2}) = \frac{1}{2\pi i} \int_{\epsilon-i\infty}^{\epsilon+i\infty} \frac{e^{\nu \tau}}{\nu} e^{- \nu \frac{k^2}{Q^2}} \dd \nu
\end{equation}

we find that 

\begin{equation}
    \Sigma(\tau,\alpha_s(Q^2)) = \int_{\epsilon - i \infty}^{\epsilon+i\infty} \tilde{J}^q(Q^2,\nu_1) \tilde{J}^q(Q^2,\nu_2) e^{\nu_1 \tau} e^{\nu_2 \tau} \dd \nu_1 \dd \nu_2
\end{equation}

where $\epsilon$ is a real positive constant to the right of all singularities of the integrand $\tilde{J}_\nu(Q^2)$ in the complex $\nu$ plane. 

An integral representation for the Laplace transform $\tilde{J}_\nu(Q^2)$ is given by

\begin{equation}\label{eq:master formula}
    \ln \tilde{J}_\nu^q(Q^2) = \int_0^1 \frac{\dd u}{u} \qty(e^{-u \nu Q^2}-1)\qty[ \int_{u^2 Q^2}^{u Q^2} \frac{1}{q^2} A\qty(\alpha_s(q^2)) \dd q^2 + \frac{1}{2} B\qty(\alpha_s(u Q^2))]
\end{equation}

with 

\begin{align*}
    A(\alpha_s) &= \sum_{n=1}^\infty \frac{A_n}{\pi^n}\alpha_s^n & B(\alpha_s) &= \sum_{n=1}^\infty \frac{B_n}{\pi^n}\alpha_s^n
\end{align*}

The integral as it is cannot be integrated, the $u$ integration may be performed using the prescription in Appendix A of \cite{Catani_2003_appendix} and
readapting the formula to the case of Laplace transform instead of Mellin transform.

In \cref{Appendix: Laplace and Mellin transform} we show that the prescription, to evaluate the large-N Mellin moments of soft-gluon
contributins at an arbitrary logarithmic accuracy, can be used for the Laplace transform as well, we can use this result to express \cref{eq:master formula} 
in an alternative representation:

The method is a generalization of the prescription to NLL accuray in \cite{CATANI19933}

\begin{equation}
    e^{-u \nu Q^2} - 1 \simeq -\Theta(u-v) \qq{with $v = \frac{N_0}{N}$}
\end{equation}

where $N_0 = e^{-\gamma_E}$, $\gamma_E = 0.5772 \dots $ being the Euler-Mascheroni constant

\begin{equation}\label{eq:master_formula alt}
    \ln \tilde{J}_\nu^q(Q^2) = - \int_{N_0/N}^{1} \frac{\dd u}{u} \qty[ \int_{u^2 Q^2}^{u Q^2} \frac{1}{q^2} A\qty(\alpha_s(q^2)) \dd q^2 + \frac{1}{2} \tilde{B}\qty(\alpha_s(u Q^2))] + \ln \tilde{C}(Q^2)
\end{equation}

\end{document} 
\makeatletter % Use \makeatletter to make '@' a letter
\def\input@path{{../}} % Define input path to parent directory
\makeatother % Restore default category code of '@'

\documentclass[../main.tex]{subfiles}

\begin{document}

\subsection{Renormalization-scale dependece}
In this section we consider renormalization-scale dependence. In principle, such scale µ should not appear in
the cross sections, as it does not correspond to any fundamental constant or kinematical scale in the problem.
The completely resummed perturbative expansion of an observable is indeed formally independent on $\mu$.
In practise, truncated perturbative expansions exhibit a residual scale dependence, because of neglected higher orders.

We start with deriving the strong coupling $\alpha_s (Q^2)$ as a function of $\alpha_s(\mu^2)$ and $\mu^2/Q^2$, to do so we expand 
the explicit \cref{eq:N4LO} in powers of $\alpha_s(\mu^2)$ and obtain:

\begin{align}
    \begin{split}\label{eq:renormalization scale dependence}
        \alpha_s(Q^2) &= \alpha_s(\mu^2) + \alpha_s^2(\mu^2) b_0 \ln(\frac{\mu^2}{Q^2}) + \alpha_s^3(\mu^2) \biggl[b_1 \ln(\frac{\mu^2}{Q^2}) + b_0^2 \ln^2\qty(\frac{\mu^2}{Q^2})\biggr] \\
        &+ \alpha_s^4 (\mu^2) \biggl[ b_2 \ln(\frac{\mu^2}{Q^2}) + \frac{5}{2} b_0 b_1 \ln^2\qty(\frac{\mu^2}{Q^2}) + b_0^3 \ln^3\qty(\frac{\mu^2}{Q^2})\biggr] \\
        &+ \frac{1}{6} \alpha_s^5(\mu^2) \biggl[6 b3 \ln(\frac{\mu^2}{Q^2}) + \qty(9 b_1^2 + 18 b_0 b_2 ) \ln^2\qty(\frac{\mu^2}{Q^2}) \\
        &+ 26 b_0^2 b_1 \ln^3\qty(\frac{\mu^2}{Q^2}) + 6 b_0^4 \ln^4\qty(\frac{\mu^2}{Q^2}) \biggr] + \order{\alpha_s^6(\mu^2)}
    \end{split}
\end{align}

for brevity we'll write 

\begin{equation}
    \alpha_s(Q^2) = \alpha_s(\mu^2) + c_1 \alpha_s^2(\mu^2) + c_2 \alpha_s^3(\mu^2) + c_3 \alpha_s^4(\mu^2) + c_4 \alpha_s^5(\mu^2) + \order{\alpha_s^6(\mu^2)} 
\end{equation}

where $c_i$ are the coefficients of the expansion above.

now we substitute  \cref{eq:renormalization scale dependence} $\lambda = b_0 \alpha_s(Q^2)$ and have :

\begin{align}
    \begin{split}
        
    \end{split}
\end{align}
\end{document}
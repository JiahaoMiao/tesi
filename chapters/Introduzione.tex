\makeatletter % Use \makeatletter to make '@' a letter
\def\input@path{{../}} % Define input path to parent directory
\makeatother % Restore default category code of '@'

\documentclass[../main.tex]{subfiles}

\begin{document}

\section{Introduction}

In this thesis we are interested in a specific aspect of theoretical high energy physics. In particular,
the topic of interest is that of electron-positron ($e^+e^-$) collisions. As is typical in physics, the equations that govern
the interactions are quite complicated and it is almost impossible to find exact solutions, so all
functions of interest are perturbatively expanded, meaning they are expanded in a power series
of a small parameter. 

When the force in question is the electromagnetic interaction, one can
simply use the fine structure constant ( or electromagnetic coupling constant ) $\alpha_{em} \sim \frac{1}{137} $.

When discussing particles that interact with the strong interaction, it is then natural to use the strong
coupling constant $ \alpha_S $.

The function we are interested in is an event shape distribution, the Thrust distribution $T$.
\begin{equation} \label{eq:Thrust}
    T = \max_{\vec{n}} \frac{\sum_i |\vec{p}_i \cdot \vec{n}|}{\sum_i |\vec{p}_i|}
\end{equation}

where the sum is over all final state particles and $\vec{n}$ is a unit vector.
It can be seen from this definition that the thrust is an infrared and collinear safe
quantity, that is, it is insensitive to the emission of zero momentum particles and to the splitting of 
one particle into two collinear ones.

It can be seen that a two-particle final state has fixed $T = 1$, consequently the thrust
distribution receives its first non-trivial contribution from three-particle final states

The lower limit on $T$ depends on the number of final-state particles.
Neglecting masses, $T_{min} = 2/3$ for three particles, corresponding to a symmetric
configuration. For four particles the minimum thrust corresponds to final-state
momenta forming the vertices of a regular tetrahedron, each making an angle
$\cos^{-1}(1/\sqrt(3))$ with respect to the thrust axis. Thus $T_{min} = 1/\sqrt(3) = 0.577$ in this
case . For more than four particles, $T_{min}$ approaches $1/2$ from above as the number of particles increases.


At large values o f $T$, however, there are terms in higher order that become enhanced by powers o f $\ln(1 - T)$.
In this kinematical region the real expansion parameter is the
large effective coupling $\alpha_s \ln^2( 1 - T)$ and therefore
any finite-order perturbative calculation cannot give an accurate evaluation o f the cross section.

These corrections lead to logarithmically enhanced terms such as $\alpha_S \log(M/qT )$, where M is
the mass of the final system and qT is the transverse momentum. For small qT , where the
bulk of events is produced, $\log(M/qT )$ can become quite large, but when $\alpha_S \log(M/qT) \sim 1$ the
perturbative expansion becomes meaningless because the neglected terms can be of the same
order of the included terms.


In order to obtain realistic cross sections, these logarithmically enhanced terms must then be
treated differently. A promising solution is the theory of resummation\cite{CATANI19933}: instead of stopping at a
certain power of $\alpha_S$ as in standard perturbation theory, we calculate an all order resummation
of the $\alpha_S \log(M/qT )$ terms which lead to a consistent exponential term, the Sudakov form
factor. It turns out that the Sudakov form factor itself is a power expansion of $\alpha_S$ but now
with no logarithmic enhancement. The first order calculation is called Leading Logarithmic (LL)
order, the next order calculation is called Next to Leading Logarithmic (NLL) and so on. In the
literature, it has been calculated up to N3 LL. We were able to calculate the next order, N4 LL,
and the main objective of this thesis is to estimate the uncertainty in this function due to the
missing higher order corrections.
In the first chapter we introduce the forces and particles of the Standard Model, also describing
the Quantum Field Theory formalism which is at the base of all the forces described in the
Standard Model. In the second chapter we then describe the process of interest and illustrate
the theorems and formulas necessary to calculate the relevant cross sections. In particular we
introduce the resummation formalism and the Sudakov form factor which is the main object of
this work. In chapter 3, we describe the calculations that lead us to obtain the Sudakov fom
factor at different logarithmic orders and finally in chapter 4 we present plots of this function and
discuss its uncertainty, first using the standard methods and then exploring different methods.

\end{document}
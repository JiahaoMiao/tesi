\makeatletter % Use \makeatletter to make '@' a letter
\def\input@path{{../}} % Define input path to parent directory
\makeatother % Restore default category code of '@'

\documentclass[../main.tex]{subfiles}

\begin{document}

\section{Introduction}

In this thesis we are interested in a specific aspect of theoretical high energy physics. In particular,
the topic of interest is that of resummantion of the Thrust event-shape distribution in electron-positron ($e^+e^-$) collisions.

As is typical in physics, the equations that govern the interactions are quite complicated and it is almost impossible to find exact solutions, so all
functions of interest are perturbatively expanded, meaning they are expanded in a power series of a small parameter. 

When the force in question is the electromagnetic interaction, one can
simply use the fine structure constant ( or electromagnetic coupling constant ) $\alpha_{em} \sim \frac{1}{137} $.

When discussing particles that interact with the strong interaction, it is then natural to use the strong
coupling constant $ \alpha_S $.

The function we are interested in is the Thrust event-shape distribution. Thrust $T$ is defined as:
\begin{equation} \label{eq:Thrust}
    T = \max_{\vec{n}} \frac{\sum_i |\vec{p}_i \cdot \vec{n}|}{\sum_i |\vec{p}_i|}
\end{equation}

where the sum is over all final state particles and $\vec{n}$ is a unit vector.
It can be seen from this definition that the thrust is an infrared and collinear safe
quantity, that is, it is insensitive to the emission of zero momentum particles and to the splitting of 
one particle into two collinear ones.

Thus the cross section 

\begin{equation}\label{eq:cross_section}
    \sigma(\tau) = \int^1_{1-\tau} \dd{T} \dv{\sigma}{T}
\end{equation}

It can be seen that a two-particle final state has fixed $T = 1$, consequently the thrust
distribution receives its first non-trivial contribution from three-particle final states

The lower limit on $T$ depends on the number of final-state particles.
Neglecting masses, $T_{min} = 2/3$ for three particles, corresponding to a symmetric
configuration. For four particles the minimum thrust corresponds to final-state
momenta forming the vertices of a regular tetrahedron, each making an angle
$\cos^{-1}(1/\sqrt(3))$ with respect to the thrust axis. Thus $T_{min} = 1/\sqrt(3) = 0.577$ in this
case . For more than four particles, $T_{min}$ approaches $1/2$ from above as the number of particles increases.

At large values o f $T$, however, there are terms in higher order that become enhanced by powers o f $\ln(1 - T)$.
In this kinematical region the real expansion parameter is the
large effective coupling $\alpha_s \ln^2( 1 - T)$ and therefore
any finite-order perturbative calculation cannot give an accurate evaluation o f the cross section.


\end{document}
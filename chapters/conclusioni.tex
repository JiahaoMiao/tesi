\documentclass[../Tesi_Jiahao_Miao_986136.tex]{subfiles}
\begin{document}

\chapter{Conclusions and Outlook}\label{ch:conclusioni}

In this thesis, we have studied the resummation of QCD large infrared logarithms for the Thrust distribution in $e^+e^-$ annihilation.
The Thrust distribution is a classical precision observable in collider physics and it allows the extraction of the strong coupling constant $\alpha_s$.

In the era of the LEP experiments(1989-2000) the calculations for certain classes of variables, such as event Thrust shape variable, 
were improved by resumming leading and next-to-leading logarithmic terms (NLL). These calculations, matched to fixed-order expressions, enlarged the
kinematic range of applicability for $\alpha_s$ extractions and reduced the systematic theoretical uncertainty. 

Our work aims to enhance these achievements by incorporating higher-order logarithmic terms in the resummation process and utilizing NNLO fixed-order calculations completed in 2007, 
which were unavailable during the LEP experiments.

Resummation tecniques are needed because in the kinematical region where the Thrust $\tau$ is close to 0, cancellations between real and virtual corrections 
are not complete due to kinematical constraints and large logarithms of the form $\alpha_s^n \log^m(\frac{1}{\tau})$ appear to spoil the convergence of the perturbative series.
Therefore, resummation of these logarithms is needed to obtain a reliable theoretical prediction for the Thrust distribution in the region of small $\tau$, 
then matching the resummed prediction with the fixed order calculation gives a prediction that is valid in a larger range of $\tau$.

The theoretical prediction is made within perturbative QCD, expanded to a finite order in the coupling constant.
The truncation of the perturbative series induces a theoretical uncertainty from omitting higher order terms.
It can be quantified by the renormalisation scale dependence of the prediction, which is vanishing for an all-order prediction.
The residual dependence on variations of the renormalisation scale is therefore an estimate of the theoretical error and we have 
observed that indeed the scale dependence for the Laplace transformed thrust distribution at N$^3$LL accuracy is significantly reduced compared to the already 
improved N$^2$LL results. However, anomalies noted in N$^4$LL results suggest potential limitations in their reliability.

Initially, we calculated the resummation functions $f_i(\lambda)$ in Laplace space, where the natural exponentiation of the logarithms occurs.
We were careful in ensuring consistency with prior findings and assembling necessary ingredients for resummation up to N$^4$LL accuracy from existing literature.

Subsequently, we performed the inverse Laplace transform of the resummed distribution following methodologies outlined by \cite{CATANI19933} and extended the results up to N$^4$LL accuracy.
This has allowed us to obtain the resummed Thrust distribution in thrust space, which already gives us a shape 
closely resembling experimental data from LEP and SLAC. The predictions and data do not match completely yet because our prediction is calculated at parton (quarks and gluons) level and does not include hadronisation effects, while
the experimental measurements are at the hadron (protons, pions, kaons, $\ldots$) level. 

To refine our theoretical predictions further, we have matched the resummed expression to the fixed-order NNLO calculations at large values of $\tau$ using the R-matching scheme.
The results presented in this thesis allow us to achieve NNLO+N$^3$LL accuracy.
This represents a step towards achieving a more precise theoretical prediction for the Thrust distribution in $e^+e^-$ annihilation and
a more precise extraction of the strong coupling constant $\alpha_s$, which is essential for the precision tests of the Standard Model and the search for new physics at the LHC.

Future endeavors will involve integrating a phenomenological model to address hadronization effects and conducting a fitting analysis with experimental data 
to accurately determine the strong coupling constant $\alpha_s$. Additionally, it is also interesting 
to investigate if the analytical Laplace inversion used in the previous literature, which we reproduced and generalised to higher orders in this thesis
gives us a reliable result. We observed that the peaks of the resummed Thrust distribution were too low compared to the experimental data, this would thus
lead to a discrepancy in the extracted value of $\alpha_s$ relative to the world average.

\end{document}

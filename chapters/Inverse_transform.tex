\makeatletter % Use \makeatletter to make '@' a letter
\def\input@path{{../}} % Define input path to parent directory
\makeatother % Restore default category code of '@'

\documentclass[../main.tex]{subfiles}

\begin{document}

In order to find the quark jet mass distribution $J^q(Q^2,k^2)$,we have to perform the inverse Laplace transform via 
the Mellin's inversion formula (or the Bromwich integral) given by the line integral: 

\begin{equation} \label{eq:inverse_laplace}
    J^q(Q^2,k^2) = \frac{1}{2\pi i} \lim_{T \to \infty}\int_{C-iT}^{C+iT} \dd \nu e^{\nu k^2} \tilde{J}^q_\nu \qty(Q^2)
\end{equation}

where C is a real number such that C is at the right of all singularities of the integrand in the complex plane and
the function $\tilde{J}^q_\nu(Q^2)$ has to be bounded on the line.

Instead of directly considering the expression in \cref{eq:inverse_laplace}, it was pointed in \cite{CATANI19933} that 
it is more convenient to work with the mass fraction $R^q(w)$, which gives the fraction of jets with masses less than $wQ^2$:

\begin{equation}\label{eq:jet mass fraction}
    R^q(w) = \int_0^\infty J^q(Q^2,k^2)\Theta(wQ^2-k^2) \dd k^2
\end{equation}

and using the integral representation of the Heaviside step function \cref{eq:heaviside_integral_representation}

\begin{equation}
    \Theta(wQ^2-k^2) = \frac{1}{2\pi i} \lim_{T \to \infty}\int_{C-iT}^{C+iT} \frac{\dd \nu}{\nu} e^{\nu (wQ^2-k^2)} 
\end{equation}

we recognize the Laplace transform of the quark jet mass distribution \cref{eq:laplace_jet_mass}

\begin{align}
    \begin{split}
        R^q(w) &= \frac{1}{2\pi i} \lim_{T \to \infty} \int_{C-iT}^{C+iT} \frac{\dd \nu}{\nu} e^{w \nu Q^2} \int_0^\infty J^q(Q^2,k^2) e^{-\nu k^2} \dd k^2 \\
        &= \frac{1}{2\pi i} \lim_{T \to \infty} \int_{C-iT}^{C+iT} \frac{\dd \nu}{\nu} e^{w \nu Q^2} \tilde{J}^q_\nu(Q^2) \\
        &= \frac{1}{2\pi i} \lim_{T \to \infty} \int_{C-iT}^{C+iT} \frac{\dd \nu}{\nu} e^{w \nu Q^2} e^{\mathcal{F}(\alpha_s,\ln(\nu Q^2))}\\
        &= \frac{1}{2\pi i} \lim_{T \to \infty} \int_{C'-iT}^{C'+iT} \frac{\dd N}{N} e^{w N} e^{\mathcal{F}(\alpha_s,\ln N)}
    \end{split}
\end{align}

where $N=\nu Q^2$ and $\mathcal{F}$ has the logarithms expansion

\begin{align}
    \begin{split}
    \mathcal{F}(\alpha_s,\ln N) &= f_1(b_0 \alpha_s \ln N) \ln N + f_2(b_0 \alpha_s \ln N) + f_3(b_0 \alpha_s \ln N) \alpha_s \\
    &+ f_4(b_0 \alpha_s \ln N) \alpha_s^2 + f_5(b_0 \alpha_s \ln N) \alpha_s^3 + \order{\alpha_s^4}
\end{split}
\end{align}

Since the function $\mathcal{F}$ in the exponent varies more slowly with N than $wN$, we can introduce the integration variable $u=wN$
so that $\ln N = \ln u +\ln \frac{1}{w} = \ln u + L$ and Taylor expand with respect to $\ln u$ around $0$, which is equivalent to expanding 
the original function $\mathcal{F}$ w.r.t $\ln N$ around $\ln N = \ln \frac{1}{w}\equiv L$:

\begin{align}
    \begin{split}\label{eq:Rw expansion}
       R^q(w) &= \frac{1}{2\pi i} \int_C \frac{\dd u}{u} e^{u} e^{\mathcal{F}(\alpha_s,\ln u + L)} \\
       &\stackrel{\text{Taylor}}{=} \int_C \frac{\dd u}{2\pi i} e^{u-\ln u} e^{\mathcal{F}(\alpha_s,L)+\sum_{n=1}^\infty \frac{\mathcal{F}^{(n)}(\alpha_s,L)}{n!}  \ln^n u}\\
       &= e^{\mathcal{F}(\alpha_s,L)} \int_C \frac{\dd u}{2\pi i} e^{u-\ln u} e^{\sum_{n=1}^\infty \frac{\mathcal{F}^{(n)}(\alpha_s,L)}{n!}  \ln^n u}
    \end{split}
\end{align}

where the integral is intended as before, along the line $C$ to the right of all singularities of the integrand, and 

\begin{equation}
    \mathcal{F}^{(n)}(\alpha_s,L) = \pdv[n]{\mathcal{F}(\alpha_s,\ln u + L )}{\ln u} \eval_{\ln u = 0}
\end{equation}

The first few derivatives are: 

\begin{align}
    \begin{split}
        \mathcal{F}^{(1)}(\alpha_s,L) &= f_1(\lambda )+\lambda  f_1'(\lambda )+\alpha_s b_0 f_2'(\lambda )+\alpha_s^2 b_0 f_3'(\lambda )+\alpha_s^3 b_0 f_4'(\lambda )\\
        &+\alpha_s^4 b_0 f_5'(\lambda )
    \end{split}
\end{align}

\begin{align}
    \begin{split}
        \mathcal{F}^{(2)}(\alpha_s,L) &= 2 \alpha_s b_0 f_1'(\lambda ) + \alpha_s b_0 \lambda  f_1''(\lambda )+\alpha_s^2 b_0^2 f_2''(\lambda )+ \alpha_s^3 b_0^2 f_3''(\lambda )\\
        &+\alpha_s^4 b_0^2 f_4''(\lambda )+ \alpha_s^5 b_0^2 f_5''(\lambda )
    \end{split}
\end{align}

\begin{align}
    \begin{split}
        \mathcal{F}^{(3)}(\alpha_s,L) &=  3 \alpha_s^2 b_0^2 f_1''(\lambda )+\alpha_s^2 b_0^2 \lambda  f_1^{(3)}(\lambda )+\alpha_s^3 b_0^3 f_2^{(3)}(\lambda )+\alpha_s^4 b_0^3 f_3^{(3)}(\lambda )\\
        &+\alpha_s^5 b_0^3 f_4^{(3)}(\lambda )+\alpha_s^6 b_0^3 f_5^{(3)}(\lambda )
    \end{split}
\end{align}

\begin{align}
    \begin{split}
        \mathcal{F}^{(4)}(\alpha_s,L) &= 4 \alpha_s^3 b_0^3 f_1^{(3)}(\lambda ) +\alpha_s^3 b_0^3 \lambda  f_1^{(4)}(\lambda )+\alpha_s^4 b_0^4 f_2^{(4)}(\lambda )+\alpha_s^5 b_0^4 f_3^{(4)}(\lambda )\\
        &+\alpha_s^6 b_0^4 f_4^{(4)}(\lambda )+\alpha_s^7 b_0^4 f_5^{(4)}(\lambda )
    \end{split}
\end{align}

After recasting the expansion presented in \cref{eq:Rw expansion} using the expression \\
$\gamma(\alpha_s,L) = f_1(\lambda) + \lambda f_1'(\lambda)$ from \cite{CATANI19933}, 
and defining $\mathcal{F}^{(1)}_{res}(\alpha_s,L) \equiv \mathcal{F}^{(1)}(\alpha_s,L) - \gamma(\alpha_s,L)$, we proceed to expand the second exponential with respect to $\ln u$ around 0, 
following the approach outlined in \cite{Aglietti:2002ew}. This yields the subsequent expansion:

\begin{align}
    \begin{split}
        R^q(w) &= e^{\mathcal{F}(\alpha_s,L)} \int_C \frac{\dd u}{2\pi i} e^{u-(1-\gamma(\alpha_s,L))\ln u} e^{\mathcal{F}^{(1)}_{res}(\alpha_s,L) \ln u +\sum_{n=2}^\infty \frac{\mathcal{F}^{(n)}(\alpha_s,L)}{n!}  \ln^n u}\\
        &= \int_C \frac{\dd u}{2\pi i} e^{u-(1-\gamma(\alpha_s,L))\ln u} \biggl( 1 + \mathcal{F}^{(1)}_{res}\ln u +\frac{1}{2}\qty( \mathcal{F}^{(2)}+(\mathcal{F}^{(1)}_{res})^2)\ln^2 u \\
        &+ \frac{1}{6}\qty(\mathcal{F}^{(3)} + 3 \mathcal{F}^{(2)}\mathcal{F}^{(1)}_{res} + (\mathcal{F}^{(1)}_{res})^3 ) \ln^3 u \\
        &+ \frac{1}{24}\qty(\mathcal{F}^{(4)} + 3(\mathcal{F}^{(2)})^2 + 4 \mathcal{F}^{(3)} \mathcal{F}^{(1)}_{res} + 6 \mathcal{F}^{(2)} (\mathcal{F}^{(1)}_{res})^2 + (\mathcal{F}^{(1)}_{res})^4 )\ln^4 u \\
        &+ \order{\ln^5 u} \biggr)
    \end{split}
\end{align}

Lastly, we utilize the following result to evaluate the integral presented in \cref{eq:jet mass fraction}.

\begin{equation}
    \int_C \frac{\dd u}{2\pi i} \ln^k u e^{u-(1-\gamma(\alpha_s,L))\ln u} = \dv[k]{\gamma} \frac{1}{\Gamma(1-\gamma(\alpha_s,L))}
\end{equation}

where $\Gamma$ is the Euler $\Gamma$-function.

\begin{align}
    \begin{split}
        R^q(w) &= \frac{e^{\mathcal{F}(\alpha_s,L)}}{\Gamma(1-\gamma)} \biggl[ 1 + \mathcal{F}^{(1)}_{res} \psi_0(1-\gamma) + \frac{1}{2}\qty( \mathcal{F}^{(2)}+(\mathcal{F}^{(1)}_{res})^2) \qty(\psi_0^2-\psi_1)(1-\gamma) \\
        &+ \frac{1}{6}\qty(\mathcal{F}^{(3)} + 3 \mathcal{F}^{(2)}\mathcal{F}^{(1)}_{res} + (\mathcal{F}^{(1)}_{res})^3 ) \qty(\psi_0^3 -3 \psi_0\psi_1+\psi_2)(1-\gamma) \\
        &+ \frac{1}{24}\qty(\mathcal{F}^{(4)} + 3(\mathcal{F}^{(2)})^2 + 4 \mathcal{F}^{(3)} \mathcal{F}^{(1)}_{res} + 6 \mathcal{F}^{(2)} (\mathcal{F}^{(1)}_{res})^2 + (\mathcal{F}^{(1)}_{res})^4 ) \\
        &\qty(\psi_0^4 -6 \psi_1 + 3\psi_1^3 +4\psi_0\psi_2 -\psi_3)(1-\gamma)+ \order{\ln^5 u} \biggr]
    \end{split}
\end{align}

where $\psi_n(z)$ are the polygamma functions, defined as:

\begin{equation}
    \psi_n (z) = \dv[n+1]{z} \ln \Gamma(z) = \dv[n]{z} \frac{\Gamma '(z)}{\Gamma(z)} = \dv[n]{z} \psi_0(z)
\end{equation}





\printbibliography

\end{document}
\makeatletter
\def\input@path{{../}}
\makeatother

\documentclass[../main.tex]{subfiles}

\begin{document}

\section{Laplace transform in the large N limit} \label{Appendix: Laplace and Mellin transform}

Using the methodology outlined in \cite{Catani_2003_appendix}, we will demonstrate that the Mellin transform prescription is also applicable to the Laplace transform.
in the large moment $\nu Q^2 = N$ limit, this fact was already known in the literature \cite{CATAN_large_chi} and we'll show it here for completeness.

We are interested in solving the following integral

\begin{equation}
    \int_0^1 \dd{z} \frac{e^{- N (1-z)} - 1}{1-z} F\qty(\alpha_s,\ln(1-z))= \int_0^1 \frac{\dd{u}}{u} \qty(e^{-u N} - 1) F\qty(\alpha_s , \ln u)
\end{equation}

Start by considering 

\begin{equation}\label{eq:appendix:laplace test f}
    I_n(N) = \int_0^1 \frac{\dd{u}}{u} \qty(e^{-u N} - 1) \ln^n(u) 
\end{equation}

the above integral can be evaluated as described in \cite{CATANI1989323}. Using the following identity

\begin{equation}
    \ln^n(u)= \lim_{\epsilon \to 0} \qty(\pdv{\epsilon})^n u^\epsilon = \lim_{\epsilon \to 0} \qty(\pdv{\epsilon})^n e^{\epsilon \ln u}
\end{equation}

to replace the logarithm term in the integrand \cref{eq:appendix:laplace test f} and straightforwardly integrate the resulting expression. We obtain

\begin{align}
    \begin{split}\label{eq:appendix:calculation}
    I_n(N) &= \lim_{\epsilon \to 0} \qty(\pdv{\epsilon})^n \int_0^1 \dd{u} \qty(e^{-u N} - 1) u^{\epsilon-1} \\
    &= \lim_{\epsilon \to 0} \qty(\pdv{\epsilon})^n \qty{-\frac{1}{\epsilon}+N^{-\epsilon} (\Gamma(\epsilon,0) - \Gamma(\epsilon,N))} \\
    &= \lim_{\epsilon \to 0} \lim_{N \to \infty} \qty(\pdv{\epsilon})^n \qty{-\frac{1}{\epsilon}+N^{-\epsilon} \Gamma(\epsilon)}+ e^{-N + \order{\qty(\frac{1}{N})^2}}\order{\frac{1}{N}} \\
    &= \lim_{\epsilon \to 0} \lim_{N \to \infty} \qty(\pdv{\epsilon})^n \qty{\frac{1}{\epsilon} (N^{-\epsilon} \epsilon \Gamma(\epsilon) -1)} + \order{\frac{e^{-N}}{N}}\\
    &= \lim_{\epsilon \to 0} \lim_{N \to \infty} \qty(\pdv{\epsilon})^n \qty{\frac{1}{\epsilon} ( e^{-\epsilon \ln N} \Gamma(1+\epsilon) -1)} + \order{\frac{e^{-N}}{N}}
    \end{split}
\end{align}

where $\Gamma(\epsilon,0) = \Gamma(\epsilon)$, $\Gamma(\epsilon,N)$ is the incomplete Gamma function and $\epsilon \Gamma(\epsilon)=\Gamma(1+\epsilon)$

\begin{equation}
    \Gamma(\epsilon,N) = \int_N^\infty \dd{t} t^{\epsilon-1} e^{-t}
\end{equation}

The last equation in \cref{eq:appendix:calculation} is the same as Eq. $(68)$ obtained in \cite{Catani_2003_appendix} for the Mellin transform. Therefore, 
we can conclude that the Mellin transform prescription is also applicable to the Laplace transform in the large $N$ limit.

Using the known expansion of the Gamma function for small $\epsilon$ 

\begin{equation}\label{eq:appendix:gamma expansion}
    \Gamma(1+\epsilon) = \exp{-\gamma_E \epsilon + \sum_{n=2}^\infty \qty(-1)^n \frac{\zeta(n) \epsilon^n}{n}}
\end{equation}

the term in curly brackets in \cref{eq:appendix:gamma expansion} can be expanded in power of $\epsilon$ and then derive. The result for $I_n(N)$ is thus a 
polynomial of degree $n+1$ in the large logarithm $\ln N$:

\begin{align}
    \begin{split}\label{eq:appendix:In expansion}    
        I_n(N) &= \frac{(-1)^n+1}{n+1} (\ln N + \gamma_E)^{n+1} + \frac{(-1)^{n-1}}{2} n \zeta(2) (\ln N + \gamma_E)^{n-1} \\
        &+ \sum_{k=0}^{n-2} a_{nk} (\ln N + \gamma_E)^k + \order{\frac{e^{-N}}{N}}
    \end{split}
\end{align}

This result can be generalized using the following formal identity:

\begin{equation}
    e^{-\epsilon \ln N} \Gamma(1+\epsilon) = \Gamma(1- \pdv{\ln N}) e^{\epsilon \ln N }
\end{equation}

then we can perform the $n$-th derivative with respect to $\epsilon$, and obtain

\begin{equation}
    I_n(N) = \Gamma\qty(1-\pdv{\ln N}) \frac{ \qty(-\ln N)^n+1}{n+1} + \order{\frac{e^{-N}}{N}}
\end{equation}

This expression can be regarded as a replacement for \cref{eq:appendix:calculation} to compute the polynomial coefficients $a_{nk}$ 
in \cref{eq:appendix:In expansion}. Moreover, by observing that 

\begin{equation}
    \frac{ \qty(-\ln N)^n+1}{n+1} = -\int_{\frac{1}{N}}^1 \dd{u} \frac{\ln^n(u)}{u}
\end{equation}

we obtain the all order generalization for of the prescriprion used in \cite{CATANI19933}:

\begin{align}
    e^{-uN}-1 &= -\Gamma\qty(1-\pdv{\ln_N}) \Theta(u-\frac{1}{N}) + \order{\frac{e^{-N}}{N}}\\
              &=  -\tilde{\Gamma}\qty(1-\pdv{\ln_N}) \Theta(u-\frac{N_0}{N}) + \order{\frac{e^{-N}}{N}}
\end{align}

where 

\begin{equation}\label{eq:appendix:gamma tilde}
    \tilde{\Gamma}\qty(1-\epsilon) \equiv e^{\gamma_E \epsilon} \Gamma(1+\epsilon) = \exp{\sum_{n=2}^\infty (-1)^n \frac{ \zeta(n) \epsilon^n}{n}}
\end{equation}

It is straightforward to show that the prescription can be applied to as follows:

\begin{equation} \label{eq:appendix:generalized prescription}
    \int_0^1 \frac{\dd{u}}{u} \qty(e^{-u N} - 1) F\qty(\alpha_s , \ln u) = -\tilde{\Gamma}\qty(1-\pdv{\ln_N}) \int_{\frac{N_0}{N}}^1 \frac{\dd{u}}{u} F\qty(\alpha_s , \ln u) + \order{\frac{e^{-N}}{N}}
\end{equation}
    
and to evaluated the ln $N$-contribution arising from the integration of anu soft-gluon function $F$ that has a generic perturbative expansion of the type

\begin{equation}
    F\qty(\alpha_s , \ln u) = \sum_{k=1}^\infty \alpha_s^k \sum_{n=0}^{2k-1} F_{kn}\ln^n u
\end{equation}

The result \cref{eq:appendix:generalized prescription} can be used to obtain \cref{eq:master_formula alt} as shown in \cite{Catani_2003_appendix}.

\section{Equivalence between resummation formulae} \label{Appendix: Coefficients}

Here i adapt the result in \cite{Catani_2003_appendix} to show the equivalence between the resummation formulae in \cref{eq:master_formula} and \cref{eq:master_formula alt}
in the case of Thrust resummation.

It is straightforward to show that equation $(90)$ in \cite{Catani_2003_appendix} becomes:

\begin{equation}
    \begin{split}
    &\int_{N_0/N}^1 \frac{\dd u}{u} \frac{1}{2}\qty(\tilde{B}(\alpha_s(uQ^2))-B(\alpha_s(uQ^2)))-\log \tilde{C}\qty( \alpha_s(\mu^2) , \frac{\mu^2}{Q^2} ) = \\
    &\Gamma_2\qty(\pdv{\log N})\qty{ A(\alpha_s(\frac{N_0}{N}Q^2)) - \frac{1}{2} \pdv{\log N}B(\alpha_s(\frac{N_0}{N}Q^2)) -2 A(\alpha_s(\frac{N_0^2}{N^2}Q^2)) }
    \end{split}
\end{equation}

Observe that using the renormalization group equation \cref{eq:RGE} and chain rule we can write the following relation:

\begin{flalign}
    \begin{split}
    \pdv{\log N} B(\alpha_s(\frac{k}{N})) &=  \pdv{B(\alpha_s)}{\alpha_s} \pdv{\alpha_s(\frac{k}{N})}{\frac{k}{N}} \pdv {\frac{k}{N}}{\log N} = - \pdv{\alpha_s(\frac{k}{N})}{\log \frac{k}{N}} \pdv{B(\alpha_s)}{\alpha_s} \\
    &= -\beta(\alpha_s) \alpha_s \pdv{B(\alpha_s)}{\alpha_s} \\
    \pdv {\log N} A(\alpha_s(\frac{k}{N^2})) &= -2 \beta(\alpha_s) \alpha_s \pdv{A(\alpha_s)}{\alpha_s} 
    \end{split}
\end{flalign}

define the differential operator $\partial(\alpha_s)$ as:

\begin{equation}
    \partial_{\alpha_s} \equiv -\beta(\alpha_s) \alpha_s \pdv{\alpha_s}
\end{equation}

Substituting the above relations in the previous equation, we obtain the equivalent of equation $(92)$ in \cite{Catani_2003_appendix}: 

\begin{equation}
    \begin{split}
    &\int_{N_0/N}^1 \frac{\dd u}{u} \frac{1}{2}\qty(\tilde{B}(\alpha_s(uQ^2))-B(\alpha_s(uQ^2)))-\log \tilde{C}\qty( \alpha_s(\mu^2) , \frac{\mu^2}{Q^2} ) = \\
    &\Gamma_2\qty(\partial_{\alpha_s}) \qty{ A(\alpha_s(\frac{N_0}{N}Q^2)) - \frac{1}{2} \partial_{\alpha_s} B(\alpha_s(\frac{N_0}{N}Q^2)) } - 2 \Gamma\qty(2\partial_{\alpha_s}) A(\alpha_s(\frac{N_0^2}{N^2}Q^2)) 
    \end{split}
\end{equation}

Now by setting $N=N_0$ or applying $\pdv{\log N}$ one obtains respectively the functions $\tilde{C}$ and  $\tilde{B}$ as functions of $A$ and $B$:

\begin{equation}\label{eq:appendix:Ctilde}
    \begin{split}
    \tilde{C}(\alpha_s) &= \exp{-\Gamma_2(\partial_{\alpha_s}) \qty[A(\alpha_s)-\frac{1}{2} B(\alpha_s)]}\\
    &- 2 \Gamma\qty(2\partial_{\alpha_s}) A(\alpha_s(\frac{N_0^2}{N^2}Q^2)) 
    \end{split}
\end{equation}
\begin{equation}\label{eq:appendix:Btilde}
    \begin{split}
    \frac{\tilde{B}(\alpha_s)}{2} &= \frac{B(\alpha_s)}{2} + \partial_{\alpha_s} \qty{\Gamma_2(\partial_{\alpha_s}) \qty[A(\alpha_s)-\frac{1}{2} \partial_{\alpha_s}B(\alpha_s)]} \\
    &- 4 \partial_{\alpha_s} \qty{\Gamma_s\qty(2\partial_{\alpha_s}) A(\alpha_s(\frac{N_0^2}{N^2}Q^2))} 
\end{split}
\end{equation}

by inserting the expansion 

\begin{align}
    \begin{split}
    \Gamma_2(\epsilon) &= -\frac{1}{2}\zeta(2) -\frac{1}{3}\zeta(3) \epsilon - \frac{9}{16}\zeta(4) \epsilon^2 - \qty(\frac{1}{6}\zeta(2)\zeta(3)+ \frac{1}{5}\zeta(5))\epsilon^3 \\
    &- \qty(\frac{1}{18}\zeta(3)^2 - \frac{61}{128}\zeta(6))\epsilon^4 + \order{\epsilon^5}   
    \end{split}
\end{align}

in \cref{eq:appendix:Btilde} and \cref{eq:appendix:Ctilde}, we can obtain the coefficients $\tilde{B}$ and $\tilde{C}$ in terms of the coefficients $A$ and $B$ up to $N^4LL$ accuracy: 

\begin{align}
    \begin{split}
    \tilde{B}\qty(\alpha_s \qty(u Q^2)) &= B\qty(\alpha_s \qty(u Q^2))  
    \end{split}
\end{align}

\begin{align}
    \begin{split}
    \log \tilde{C}\qty(\alpha_s(mu^2), \frac{\mu^2}{Q^2}) &= \frac{A_1}{\pi } (-\zeta (2)-1) \alpha_s + \biggl(\frac{-2 A_2 \zeta (2)-2 A_2+\pi  b_0 B_1}{2 \pi ^2}\\
    &+\frac{A_1 b_0 \left(3 \zeta (2) \log \left(\frac{\mu ^2}{Q^2}\right)+3 \log \left(\frac{\mu ^2}{Q^2}\right) -4 \zeta (3)\right)}{3 \pi }\biggr)\alpha_s^2\\
    &+\biggl(\frac{A_1}{3 \pi } \Bigl(-27 b_0^2 \zeta (4)-3 b_0^2 \zeta (2) \log ^2\left(\frac{\mu ^2}{Q^2}\right)-3 b_0^2 \log ^2\left(\frac{\mu ^2}{Q^2}\right)\\
    &+8 b_0^2 \zeta (3) \log \left(\frac{\mu ^2}{Q^2}\right)+3 b_1 \zeta (2) \log \left(\frac{\mu ^2}{Q^2}\right)+3 b_1 \log \left(\frac{\mu ^2}{Q^2}\right)\\
    &-4 b_1 \zeta (3)\Bigr)-\frac{1}{6 \pi ^3}\Bigl(-12 \pi  A_2 b_0 \zeta (2) \log \left(\frac{\mu ^2}{Q^2}\right)\\
    &-12 \pi  A_2 b_0 \log \left(\frac{\mu ^2}{Q^2}\right)+16 \pi  A_2 b_0 \zeta (3)+6 A_3 \zeta (2)+6 A_3\\
    &+6 \pi ^2 b_0^2 B_1 \log \left(\frac{\mu ^2}{Q^2}\right)-6 \pi  b_0 B_2 -3 \pi ^2 b_1 B_1\Bigr)\biggr)\alpha_s^3 + \order{\alpha_s^4}
    \end{split}
\end{align}
We note that $\tilde{B}$ corrects the $B$ terms so it has to be expanded up to $\alpha_s^4$ to achieve $N^4LL$ accuracy while
$\ln \tilde{C}$ corrects the $f_i$ functions so they have to be expanded up to $\alpha_s^3$ to achieve $N^4LL$ accuracy, 
these corrections are necessary only for NNLL accuracy and beyond, consistent with the results in \cite{CATANI19933}.

\end{document}
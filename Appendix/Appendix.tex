\makeatletter
\def\input@path{{../}}
\makeatother

\documentclass[../main.tex]{subfiles}

\begin{document}

\section{Laplace and Mellin transform in the large N limit} \label{Appendix: Laplace and Mellin transform}

Using the methodology outlined in \cite{Catani_2003_appendix}, we will demonstrate that the Mellin transform prescription is also applicable to the Laplace transform.
in the large moment $\nu Q^2 = N$ limit, this fact was already known in the literature \cite{CATAN_large_chi} and we'll show it here for completeness.

We are interested in solving the following integral

\begin{equation}
    \int_0^1 \dd{z} \frac{e^{- N (1-z)} - 1}{1-z} F\qty(\alpha_s,\ln(1-z))= \int_0^1 \frac{\dd{u}}{u} \qty(e^{-u N} - 1) F\qty(\alpha_s , \ln u)
\end{equation}

Start by considering 

\begin{equation}\label{eq:appendix:laplace test f}
    I_n(N) = \int_0^1 \frac{\dd{u}}{u} \qty(e^{-u N} - 1) \ln^n(u) 
\end{equation}

the above integral can be evaluated as described in \cite{CATANI1989323}. Using the following identity

\begin{equation}
    \ln^n(u)= \lim_{\epsilon \to 0} \qty(\pdv{\epsilon})^n u^\epsilon = \lim_{\epsilon \to 0} \qty(\pdv{\epsilon})^n e^{\epsilon \ln u}
\end{equation}

to replace the logarithm term in the integrand \cref{eq:appendix:laplace test f} and straightforwardly integrate the resulting expression. We obtain

\begin{align}
    \begin{split}\label{eq:appendix:calculation}
    I_n(N) &= \lim_{\epsilon \to 0} \qty(\pdv{\epsilon})^n \int_0^1 \dd{u} \qty(e^{-u N} - 1) u^{\epsilon-1} \\
    &= \lim_{\epsilon \to 0} \qty(\pdv{\epsilon})^n \qty{-\frac{1}{\epsilon}+N^{-\epsilon} (\Gamma(\epsilon,0) - \Gamma(\epsilon,N))} \\
    &= \lim_{\epsilon \to 0} \lim_{N \to \infty} \qty(\pdv{\epsilon})^n \qty{-\frac{1}{\epsilon}+N^{-\epsilon} \Gamma(\epsilon)}+ e^{-N + \order{\qty(\frac{1}{N})^2}}\order{\frac{1}{N}} \\
    &= \lim_{\epsilon \to 0} \lim_{N \to \infty} \qty(\pdv{\epsilon})^n \qty{\frac{1}{\epsilon} (N^{-\epsilon} \epsilon \Gamma(\epsilon) -1)} + \order{\frac{e^{-N}}{N}}\\
    &= \lim_{\epsilon \to 0} \lim_{N \to \infty} \qty(\pdv{\epsilon})^n \qty{\frac{1}{\epsilon} ( e^{-\epsilon \ln N} \Gamma(1+\epsilon) -1)} + \order{\frac{e^{-N}}{N}}
    \end{split}
\end{align}

where $\Gamma(\epsilon,0) = \Gamma(\epsilon)$, $\Gamma(\epsilon,N)$ is the incomplete Gamma function and $\epsilon \Gamma(\epsilon)=\Gamma(1+\epsilon)$

\begin{equation}
    \Gamma(\epsilon,N) = \int_N^\infty \dd{t} t^{\epsilon-1} e^{-t}
\end{equation}

The last equation in \cref{eq:appendix:calculation} is the same as Eq. $(68)$ obtained in \cite{Catani_2003_appendix} for the Mellin transform. Therefore, 
we can conclude that the Mellin transform prescription is also applicable to the Laplace transform in the large $N$ limit.

Using the known expansion of the Gamma function for small $\epsilon$ 

\begin{equation}\label{eq:appendix:gamma expansion}
    \Gamma(1+\epsilon) = \exp{-\gamma_E \epsilon + \sum_{n=2}^\infty \qty(-1)^n \frac{\zeta(n) \epsilon^n}{n}}
\end{equation}

the term in curly brackets in \cref{eq:appendix:gamma expansion} can be expanded in power of $\epsilon$ and then derive. The result for $I_n(N)$ is thus a 
polynomial of degree $n+1$ in the large logarithm $\ln N$:

\begin{align}
    \begin{split}\label{eq:appendix:In expansion}    
        I_n(N) &= \frac{(-1)^n+1}{n+1} (\ln N + \gamma_E)^{n+1} + \frac{(-1)^{n-1}}{2} n \zeta(2) (\ln N + \gamma_E)^{n-1} \\
        &+ \sum_{k=0}^{n-2} a_{nk} (\ln N + \gamma_E)^k + \order{\frac{e^{-N}}{N}}
    \end{split}
\end{align}

This result can be generalized using the following formal identity:

\begin{equation}
    e^{-\epsilon \ln N} \Gamma(1+\epsilon) = \Gamma(1- \pdv{\ln N}) e^{\epsilon \ln N }
\end{equation}

then we can perform the $n$-th derivative with respect to $\epsilon$, and obtain

\begin{equation}
    I_n(N) = \Gamma\qty(1-\pdv{\ln N}) \frac{ \qty(-\ln N)^n+1}{n+1} + \order{\frac{e^{-N}}{N}}
\end{equation}

This expression can be regarded as a replacement for \cref{eq:appendix:calculation} to compute the polynomial coefficients $a_{nk}$ 
in \cref{eq:appendix:In expansion}. Moreover, by observing that 

\begin{equation}
    \frac{ \qty(-\ln N)^n+1}{n+1} = -\int_{\frac{1}{N}}^1 \dd{u} \frac{\ln^n(u)}{u}
\end{equation}

we obtain the all order generalization for of the prescriprion used in \cite{CATANI19933}:

\begin{align}
    e^{-uN}-1 &= -\Gamma\qty(1-\pdv{\ln_N}) \Theta(u-\frac{1}{N}) + \order{\frac{e^{-N}}{N}}\\
              &=  -\tilde{\Gamma}\qty(1-\pdv{\ln_N}) \Theta(u-\frac{N_0}{N}) + \order{\frac{e^{-N}}{N}}
\end{align}

where 

\begin{equation}
    \tilde{\Gamma}\qty(1-\epsilon) \equiv e^{\gamma_E \epsilon} \Gamma(1+\epsilon) = \exp{\sum_{n=2}^\infty (-1)^n \frac{ \zeta(n) \epsilon^n}{n}}
\end{equation}

It is straightforward to show that the prescription can be applied to as follows:

\begin{equation} \label{eq:appendix:generalized prescription}
    \int_0^1 \frac{\dd{u}}{u} \qty(e^{-u N} - 1) F\qty(\alpha_s , \ln u) = -\tilde{\Gamma}\qty(1-\pdv{\ln_N}) \int_{\frac{N_0}{N}}^1 \frac{\dd{u}}{u} F\qty(\alpha_s , \ln u) + \order{\frac{e^{-N}}{N}}
\end{equation}
    
and to evaluated the ln $N$-contribution arising from the integration of anu soft-gluon function $F$ that has a generic perturbative expansion of the type

\begin{equation}
    F\qty(\alpha_s , \ln u) = \sum_{k=1}^\infty \alpha_s^k \sum_{n=0}^{2k-1} F_{kn}\ln^n u
\end{equation}

The result \cref{eq:appendix:generalized prescription} can be used to obtain \cref{eq:master_formula alt} as shown in \cite{Catani_2003_appendix}.

The coefficients $\tilde{B}$ and $\tilde{C}$ are related to the coefficients $A$ and $B$ in the following way:

\begin{align}
\tilde{B}(\alpha_s) &= B(\alpha_s) + 4 \partial_\alpha \Gamma_2(\partial_\alpha) \qty[A(\alpha_s)-\frac{1}{4} B(\alpha_s)] \label{eq:appendix:Btilde}\\ 
\tilde{C}(\alpha_s) &= \exp{-4\Gamma_2(\partial_\alpha) \qty[A(\alpha_s1-\frac{1}{4} B(\alpha_s))]} \label{eq:appendix:Ctilde}
\end{align}

where 

\begin{align}
    \Gamma_2(\partial_\alpha) &= \frac{1}{\epsilon^2}\qty[1-e^{-\gamma_E \epsilon \Gamma(1-\epsilon)}] = \frac{1}{\epsilon^2}\qty{1-\exp[\sum_{n=2}^\infty \frac{\zeta(n)}{n}\epsilon^n] }\\
    \partial_\alpha &\equiv -2\beta(\alpha_s)\alpha_s \pdv{\alpha_s}
\end{align}

$\beta(\alpha_s)$ is the QCD beta function \cref{eq:RGE}

by inserting the expansion 

\begin{align}
    \begin{split}
    \Gamma_2(\epsilon) &= \frac{1}{2}\zeta(2) -\frac{1}{3}\zeta(3) \epsilon - \frac{9}{16}\zeta(4) \epsilon^2 - \qty(\frac{1}{6}\zeta(2)\zeta(3)+ \frac{1}{5}\zeta(5))\epsilon^3 \\
    &- \qty(\frac{1}{18}\zeta(3)^2 - \frac{61}{128}\zeta(6))\epsilon^4 + \order{\epsilon^5}   
    \end{split}
\end{align}

in \cref{eq:appendix:Btilde} and \cref{eq:appendix:Ctilde}, we can obtain the coefficients $\tilde{B}$ and $\tilde{C}$ in terms of the coefficients $A$ and $B$ up to $N^4LL$ accuracy: 

\begin{align}
    \begin{split}
    \tilde{B}\qty(\alpha_s \qty(u Q^2)) - B\qty(\alpha_s \qty(u Q^2)) &= -\frac{4 A_1 b_0}{\pi} (\zeta(2)) \alpha_s^2 + \Bigl(-\frac{8 A_2 b_0}{\pi^2} \zeta(2) + \frac{4 b_0^2 B_1}{\pi} \zeta(2) \\
    &- \frac{4 A_1}{3\pi}(8 b_0^2 \zeta(3) + 3 b_1 \zeta(2))\Bigr) \alpha_s^3 + \biggl(-\frac{12 A_3 b_0}{\pi^3} \zeta(2) \\
    &+ \frac{12 b_0^2 B_2}{\pi^2} \zeta(2) - \frac{8 A_2}{\pi^2} (4 b_0^2 \zeta(3) + b_1 \zeta(2)) \\
    &+ \frac{B_1}{3 \pi}(48 b_0^3 \zeta(3) +  30 b_0 b_1 \zeta(2)) - \frac{4 A_1}{(3 \pi)} (20 b_0 b_1 \zeta(3) \\
    &+ 3 b_2 \zeta(2) + 81 b_0^3 \zeta(4)) \biggr) \alpha_s^4 + \order{\alpha_s^5}
    \end{split}
\end{align}

% \biggl( -\frac{16 A_4 b_0}{\pi^4} \zeta(2) + \frac{24 b_0^2 B_3}{\pi^3} \zeta(2) \\
% &- \frac{4 A_3}{\pi^3} (16 b_0^2 \zeta(3) + 3 b_1 \zeta(2)) - \frac{2 B_2}{15 \pi^2} (-480 b_0^3 \zeta(3) - 210 b_0 b_1 \zeta(2)) \\
% &- \frac{8 A_2}{3 \pi^2} (28 b_0 b_1 \zeta(3) + 3 b_2 \zeta(2) + 162 b_0^3 \zeta(4)) - \frac{2 B_1}{15 \pi} (-520 b_0^2 b_1 \zeta(3) \\
% &- 45 b_1^2 \zeta(2) - 90 b_0 b_2 \zeta(2) - 1620 b_0^4 \zeta(4)) - \frac{2 A_1}{15 \pi} (120 b_1^2 \zeta(3) \\
% &+ 240 b_0 b_2 \zeta(3) + 2304 b_0^4 \zeta(5) + 30 b_3 \zeta(2) + 1920 b_0^4 \zeta(3) \zeta(2) \\ 
% &+ 3510 b_0^2 b_1 \zeta(4))\biggr)\alpha_s^5 + \order{\alpha_s^6}

\begin{align}
    \begin{split}
    \ln \tilde{C}\qty(\alpha_s \qty(Q^2, \frac{\mu^2}{Q^2})) &= \frac{2 A_1 \zeta(2)}{\pi} \alpha_s + \biggl(\frac{2 A_2}{\pi^2} \zeta(2) - \frac{b_0 B_1}{\pi} \zeta(2) + \frac{A_1}{3\pi}(8 b_0 \zeta(3) \\
    &- 6 b_0 \ln\qty[\frac{\mu^2}{Q^2}] \zeta(2))\biggr)\alpha_s^2 + \biggl( \frac{2 A_3}{\pi^3} \zeta(2) - \frac{2 b_0 B2}{\pi^2} \zeta(2) \\
    &+ \frac{A_2}{3 \pi^2} (16 b_0 \zeta(3) - 12 b_0 \ln\qty[\frac{\mu^2}{Q^2}] \zeta(2)) + \frac{B_1}{3 \pi} \Bigl(-8 b_0^2 \zeta(3) \\
    &- 3 b_1 \zeta(2) + 6 b_0^2 \ln\qty[\frac{\mu^2}{Q^2}] \zeta(2)\Bigr) + \frac{A_1}{3 \pi} \Bigl(8 b_1 \zeta(3) \\
    &+  6 b_0^2 \ln^2\qty[\frac{\mu^2}{Q^2}] \zeta(2) - 2 \ln\qty[\frac{\mu^2}{Q^2}] (8 b_0^2 \zeta(3) + 3 b_1 \zeta(2)) \\
    &+ 54 b_0^2 \zeta(4)\Bigr) \biggr)\alpha_s^3 + \order{\alpha_s^4}
    \end{split}
\end{align}

We note that $\tilde{B}$ corrects the $B$ terms so it has to be expanded up to $\alpha_s^4$ to achieve $N^4LL$ accuracy while
$\ln \tilde{C}$ corrects the $f_i$ functions so they have to be expanded up to $\alpha_s^3$ to achieve $N^4LL$ accuracy, 
these corrections are necessary only for NNLL accuracy and beyond, consistent with the results in \cite{CATANI19933}.

\end{document}
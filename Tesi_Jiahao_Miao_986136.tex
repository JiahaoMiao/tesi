\documentclass[10pt,english,twoside]{book}

\usepackage{style/relazione}
\usepackage{style/mystyle}

% Adding the path for graphics
\graphicspath{{../figures/}}

% Setting the path for input files
\makeatletter
\def\input@path{{../}}
\makeatother

\addbibresource{bibliography/biblio.bib} % The file containing the bibliography

% BACHELOR DEGREE COURSE
\def\myCDL{Corso di Laurea triennale in Fisica}

% TITLE:
\def\myTitle{Resummation of QCD large infrared logarithms \\for Thrust distribution in Electron-Positron Annihilation} 

% AUTHOR:
\def\myName{Jiahao Miao}
\def\myMat{Matr. Nr. 986136}
 
\def\myRefereeA{Giancarlo Ferrera}
\def\myRefereeB{Wan-Li Ju}
 
% ANNO ACCADEMICO
\def\myYY{2023-2024}
 
% This command introduces a list of figures after the table of contents (optional)
\figurespagetrue

% This command introduces a list of tables after the table of contents (optional)
\tablespagetrue


\begin{document}

% frontispiece with logo, title, author, referee, etc.
\frontispiece

\beforepreface % romans page numbering, sets pagestyle to empty
\prefacesection{Abstract} %Abstract

This thesis focuses on the Thrust event-shape distribution in electron-positron 
annihilation, a classical precision collider observable which can be measured
very accurately and provides an ideal proving ground for testing our understanding of strong interactions.
Our particular focus lies on the back-to-back region, where we employ all-order resummation techniques to address logarithmically enhanced 
contributions within QCD perturbation theory. Our calculations extend the pioneering work of Catani et al \cite{CATANI19933}
in this field, aiming to achieve next-to-next-to-next-to-leading logarithmic (N$^3$LL) accuracy 
and then consistently match with the known fixed-order results up to next-to-next-to-leading order (NNLO).

Our approach involves calculating the resummation coefficients in Laplace space where natural exponentiation of the logarithmic contributions occurs,
perform the inverse Laplace transform analytically up to N$^4$LL accuracy to obtain the resummed distribution in thrust space and then
match the resummed expression to the fixed-order NNLO calculations at large values of $\tau$ using the R-matching scheme.
We compare each step with experimental results, even though the experimental data are measured on hadron level while our prediction is calculated at parton level.
Non-perturbative effects affecting the peak Thrust distribution are not included;
a phenomenological model describing hadronization effects would enhance the agreement with experimental data, aiding in the 
extraction of the strong coupling constant $\alpha_s$.
The results presented in this thesis are a step towards achieving a more precise theoretical prediction for the Thrust distribution in $e^+e^-$ annihilation and
a more precise extraction of the strong coupling constant $\alpha_s$, which is essential for the precision tests of the Standard Model and the search for new physics at the LHC. 

% \afterpreface % table of contents, tables and figures



\tableofcontents
\pagenumbering{gobble}


\allowdisplaybreaks % To allow page breaks in the middle of equations
\subfile{chapters/Introduzione}

\subfile{chapters/Resummation}

\subfile{chapters/Strong_Coupling}

\subfile{chapters/f_coeff}

\subfile{chapters/scale_dependence}

\subfile{chapters/Inverse_transform}

\subfile{chapters/Matching}

\subfile{chapters/conclusioni}
\appendix

\subfile{Appendix/Appendix}
\subfile{Appendix/Ingredients}
% bibliography 
\printbibliography
\end{document}
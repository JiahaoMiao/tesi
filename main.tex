\documentclass[10pt,english]{article}

\usepackage{style/relazione}
\usepackage{style/mystyle}


\addbibresource{bibliography/biblio.bib} % The file containing the bibliography

% BACHELOR DEGREE COURSE
\def\myCDL{Corso di Laurea triennale in Fisica}

% TITLE:
\def\myTitle{Thrust distribution in \lowercase{$e^+e^-$} annihilation} 

% AUTHOR:
\def\myName{Miao Jiahao}
\def\myMat{Matr. Nr. 936136}
 
\def\myRefereeA{Giancarlo Ferrera}
\def\myRefereeB{Wan-Li Ju}
 
% ANNO ACCADEMICO
\def\myYY{2023-2024}

% This command introduces a list of figures after the table of contents (optional)
\figurespagetrue

% This command introduces a list of tables after the table of contents (optional)
\tablespagetrue


\begin{document}

% frontispiece with logo, title, author, referee, etc.
\frontispiece

\beforepreface % romans page numbering, sets pagestyle to empty
\prefacesection{Abstract} %Abstract

This thesis focuses on the Thrust event-shape distribution in electron-positron 
annihilation, a classical precision QCD observable. We focus on the back-to-back region
and perform all-order resummation to address logarithmically enhanced contributions 
within QCD perturbation theory. Our calculations extend the pioneering work of Catani in this field,
aiming to achieve next-to-next-to-next-to-leading logarithmic (N$^3$LL) accuracy 
and then consistently combine with the known fixed-order results up to next-to-next-to-leading order 
(NNLO). Almost all the calculations are done analytically using the software Mathematica.

% \afterpreface % table of contents, tables and figures

\tableofcontents
\pagenumbering{arabic}
\setcounter{page}{1}

\subfile{chapters/Introduzione} 

\subfile{chapters/Resummation}

\subfile{chapters/Strong_Coupling}

\subfile{chapters/f_coeff}

\subfile{chapters/Inverse_transform}

\appendix
\subfile{Appendix/Appendix}

% bibliography 
\printbibliography

\end{document}
\documentclass[12pt,english]{article}

\usepackage{style/relazione}
\usepackage{style/mystyle}

\addbibresource{bibliography/biblio.bib} % The file containing the bibliography

% BACHELOR DEGREE COURSE
\def\myCDL{Corso di Laurea triennale in Fisica}

% TITLE:
\def\myTitle{Thrust distribution in \lowercase{$e^+e^-$} annihilation} 

% AUTHOR:
\def\myName{Miao Jiahao}
\def\myMat{Matr. Nr. 936136}
 
\def\myRefereeA{Giancarlo Ferrera}
\def\myRefereeB{Wan-Li Ju}
 
% ANNO ACCADEMICO
\def\myYY{2023-2024}

% This command introduces a list of figures after the table of contents (optional)
\figurespagetrue

% This command introduces a list of tables after the table of contents (optional)
\tablespagetrue


\begin{document}

% frontispiece with logo, title, author, referee, etc.
\frontispiece

\beforepreface % romans page numbering, sets pagestyle to plain
\prefacesection{Abstract} %Abstract
\begin{abstract}
    This thesis focuses on the Thrust event-shape distribution in electron-positron annihilation, 
    a classical precision QCD observable. We focus on the back-to-back region and perform all-order resummation 
    to address logarithmically enhanced contributions within QCD perturbation theory. Our calculations extend the pioneering work of Catani in this field, 
    aiming to achieve next-to-next-to-next-to-leading logarithmic (N^3LL) accuracy and then consistently combine with the known fixed-order results up to next-to-next-to-leading order
    (NNLO). Almost all the calculations are done analytically using the software Mathematica.
\end{abstract} 
\afterpreface % table of contents, tables and figures, page numbering arabic


\subfile{chapters/Introduzione} 

\subfile{chapters/Resummation}

\subfile{chapters/Strong_Coupling}

\subfile{chapters/f_coeff}

\subfile{chapters/scale_dependence}

\subfile{chapters/Inverse_transform}

\appendix
\subfile{Appendix/Appendix}

% bibliography 
\printbibliography

\end{document}